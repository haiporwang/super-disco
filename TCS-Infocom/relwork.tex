A coflow is a flow set transferring data between two stages \cite{chowdhury2012coflow}. For a better application performance, we need to reduce the coflow completion time (CCT). Many works have been done in the classical networks. However, in optical circuit switched networks, how to reduce the CCT remains an open problem. There are two important ways to reduce the CCT. One is coflow scheduling, the other is scheduling the circuits in the OCS network. In the part, we will introduce the corresponding works respectively.

Many works have studied the coflow scheduling in the data center networks. The author of \cite{chowdhury2012coflow} summarized the traffic patterns and explicitly proposed the concept of coflow. After that, researches like \cite{chowdhury2014efficient} and \cite{dogar2014decentralized} started to apply the coflow concept in their optimizations. All of the above works focused on when to forward the flows while neglecting the routing part. Only the author of \cite{zhao2015rapier} added the routing problem for coflows in the DCNs. However, the classical networks are packet switched networks. The OCS networks are circuits switched networks \cite{huang2016sunflow}. The coflow scheduling for packet switched networks are not suitable for circuits switched networks because optical switches have port constraints and the lightpath is not always available. To the best of our knowledge, no research has studied the coflow scheduling (including routing part) in the OCS network ever before. Most of the researches about optimizing the CCT in the OCS network focus on the circuits scheduling.

Another important method is circuits scheduling. Because the optical circuit switches have port constraint, a natural topic is scheduling the circuits to minimize the CCT of the coflow. The researches like \cite{farrington2012hunting} and \cite{porter2013integrating} scheduled the circuits with the classic BvN matrix decomposition algorithm \cite{birkhoff1946tres}. To use BvN, they would pre-process the demand matrixto meet the inpt assumption of BvN and decompose the pre-processed matrix into assignments and weights for each assignment. The author of \cite{liu2015scheduling} added dummy demand to the demand matrix in it's pre-process step. The above researches are based on the \emph{all-stop} model, which assumes communication stops on all optical circuits during reconfiguration. It's unnecessary because circuits unchanged are not impacted and keep serving traffic \cite{huang2016sunflow}. Besides, the above researches all focused on the scheduling in a single switch not the whole optical network. The author of \cite{wang2015end} focused on the scheduling for optical networks based on the \emph{all-stop} model. Sunflow \cite{huang2016sunflow} studied the circuits scheduling based on the \emph{not-all-stop} model, but it's scheduling is a single optical circuit switch scheduling not the circuits scheduling of a OCS network. To the best of our knowledge, no research has studied the circuits scheduling of the whole OCS network.

Coflow scheduling and circuits scheduling both influence the CCT. The coordination of coflow scheduling and circuits scheduling is complicated but necessary.  Our paper will integrate these two ways to minimize the CCT.



