
\begin{IEEEproof}
For the CRS problem, the OCS network scheduling scheme is fixed. We consider a special case that the coflow only has the flow set $f_{i,j}=\{f_{i,j}^1,f_{i,j}^2,...\}$ from source ToR switch $i$ to destination ToR switch $j$. There is a lightpath set $P_{i,j}$ which includes multiple lightpaths $p\in P_{i,j}$ connecting ToR switch $i$ and ToR switch $j$. Each lightpath has active periods. What's more, there is an interval T between two active periods. Let's consider a special case that interval T of each lightpath is 0 and each lightpath starts at the initial moment. Thus, each light path is active all the time, our objective is to minimize the completion time. Then our problem because the \emph{multi machine scheduling} problem, which is known to be NP-hard \cite{mosheiov1998multi}. The special case of our problem is NP-hard, so our problem is NP-hard too.
\end{IEEEproof}

\begin{IEEEproof}
For the CS problem, Similar to the proof of \ref{CRS:nphard}. We consider a special case that the OCS network scheduling scheme is fixed and the coflow only has the flow set $f_{i,j}=\{f_{i,j}^1,f_{i,j}^2,...\}$ from source ToR switch $i$ to destination ToR switch $j$. There is a lightpath set $P_{i,j}$ which includes multiple lightpaths $p\in P_{i,j}$ connecting ToR switch $i$ and ToR switch $j$. Each lightpath has active periods. We consider an special case that the active periods of each lightpath is infinite and each lightpath starts at the initial moment. Thus the flows won't be interrupted (active period is long enough) and the first active period of each lightpath is enough for the flows scheduling. Under this special case, our problem becomes the \emph{multi machine scheduling} problem, which is known to be NP-hard \cite{mosheiov1998multi}. The special case of our problem is NP-hard, so we have proven the NP-hardness of our problem.
\end{IEEEproof}
